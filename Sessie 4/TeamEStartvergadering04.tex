\documentclass{article}
\usepackage[dutch]{babel}
\usepackage{hyperref}
\usepackage{graphicx}


\title{Startvergadering ML sessie 4}
\author{Team $\exists$uler \and
	\textit{Daan, Marie, Zeineb, Florian, Vincent, Jasper, Lasha, Younes}}
\date{Vrijdag \today}

\begin{document}
	
	\maketitle
	
	\section*{Overlopen van de oefeningen}
	
	Vorige week hadden we afgesproken om elk nog zo veel mogelijk oefeningen af te werken en bovendien elkaars oefeningen te controleren. We hebben alle oefeningen geüpload naar de gedeelde folder \textit{Oefeningen} op OneDrive en het bijhorende tabblad in de KanBan werd ook aangevuld.
	
	Vincent heeft nog een aantal oefeningen afgewerkt en gecontroleerd. ...
	
	\section*{Overlopen van de machine learning technieken}
	
	Vorige sessie hadden we afgesproken om ons allemaal wat te verdiepen in 1 of meerdere machine learning technieken en hier een kort tekstje over te schrijven. Deze tekst zouden we dan allemaal uploaden naar de OneDrive-map \textit{ML Technieken}. Op die manier zouden we de beste 3 technieken kunnen selecteren, waarvan we er eentje zullen moeten toepassen, naast de verplichte KNN en lineaire regressie.
	
	Hieronder een overzicht van de verschillende technieken per teamlid:
	
	\begin{description}
		\item[Daan] Random Forest
		\item[Marie] Random Forest
		\item[Vincent] K-means clustering, SVM's en logistische regressie
		\item[Younes] Principal Component Analysis
		\item[Zeineb] KNN (moesten we sowieso doen)
		\item[Florian] Neurale Netwerken
		\item[Lasha] Neurale Netwerken
		\item[Jasper] Local Outlier Factor
	\end{description}	
	

\end{document}