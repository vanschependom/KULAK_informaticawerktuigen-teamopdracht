\documentclass[TeamE-eindrapport]{subfiles}

\begin{document}
	
	\chapter{Supervised Learning}
	
	\section{Supervised vs Unsupervised}
	
	\emph{Machine Learning} wordt meestal gesplitst onder twee categoriën. \emph{Supervised Learning} en \emph{Unsupervised Learning}. Het grootste verschil tussen de twee, is dat \emph{Supervised Learning} in tegenstelling tot \emph{Unsupervised Learning} data met labels gebruikt.
	Bij \emph{Supervised Learning} bestaat data uit parameters en labels. Men traint een \emph{Supervised Learning} techniek op dergelijke data waardoor men vervolgens zal kunnen de labels voorspellen voor nieuwe gegeven waarden voor de parameters.
	Bij \emph{Unsupervised Learning} zal de data enkel uit parameters bestaan. Een  \emph{Unsupervised Learning} techniek zal zelf patronen/structuren vinden in de data en vervolgens zelf labels uitvinden en geven aan de tuples van data.
	
	\section{Vormen van Supervised Learning}
	\emph{Supervised Learning} en \emph{Unsupervised Learning} kan nog verder gesplitst worden in categoriën; de twee voornaamste vormen van \emph{Supervised Learning} zijn regressie en classificatie. Bij regressie bestaan de labels altijd uit getallen, waarbij de labels oneindige verschillende waarden kunnen aannemen. Dit staat in tegenstelling met classificatie waar de labels enkel een telbaar aantal waarden kunnen aannemen, dit kunnen getallen zijn wat quantitatieve classificatie genoemd wordt of ze kunnen ook woorden zijn wat qualitatieve classificatie genoemd wordt.
	Classificatie zal later in wat meer detail uitgelegd worden.
	De verschillende vormen van \emph{Unsupervised Learning} ligt buiten de scope van dit rapport.
	
\end{document}