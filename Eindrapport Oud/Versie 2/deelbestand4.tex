\documentclass[TeamE-eindrapport]{subfiles}
\usepackage{graphicx}
\usepackage{amsmath}
\begin{document}
	
	\section{De Schatter}
	omdat we meestal niet kunnen bepalen wat precies de variantie is van data hebben we nood aan een functie die de variantie schat, dit is de schatter.
	een veel gebruikte schatter is het steekproefgemiddelde. Deze word berekent met de formule:
	\begin{equation}
		\bar{X_n} = \fraq{1}{n} \sum_{i=1}^{n} X_i
	\end{equation}
	Waarbij \bar{X_n} de gemiddelde Stochast, n het aantal datapunten, en X_i het i-de datapunt. 
	Een schatter zonder bias is niet altijd beter. 
	
	
	
	\section{accuraatheid}
	om de accuraatheid te bepalen van ons model splitsen we onze data in 3 delen namelijk: trainingsdata, validatiedata en testdata.
	vaak wordt gesplits in: 50\% trainingsdata, 25\% validatiedata, 25\testdata. Nadat het model is getraind met de trainingsdata, wordt het
	gevalideerd via de validatiedata en eventueel opnieuw getrained als de MSE te groot is. De testdata is voor de gebruiker om te zien of het model werkt
	voor nog nooit geziene data
	
	\section{Bias}
	Bias kunnen we in woorden beschrijven als de gemiddelde fout op een voorspellde waarde. Bijvoorbeeld als de
	verwachte waarde 13 en de werkelijke waarde 65 zal de bais veel groter zijn dan wanneer de werkelijke waarde 14 is.
	Als we de bias wiskundig willen bereken doen we dit met de volgende formule:
	\begin{equation}
		bias( \hat{\theta}) = E( \hat{\theta}) − \theta
	\end{equation}
	waarbij \(\theta\) gelijk is aan de variantie en\( E( \hat{\theta})\) gelijk is aan de verwachte waarde van de schatter.
	Stel dat de bias nul is dan zal \(\theta = E( \hat{\theta})\). 
	Dan noemen we dit een onvertekende schatter (zie fig x,x+1). Een lage bias is niet altijd beter.
	Een lage bias kan lijden tot overfitting waarbij het model te veel is aangepast
	aan de trainingsdata en er dus slechte voorspellingen worden gedaan voor de validatiedata en de testdata. De bias kan
	ook worden gebruikt in de Formule voor de Mean-Squared Error(MSE) die bepaald hoe algemeen sterk het model is. De term Bias kan ook slaan op
	bepaalde stigmas die het model hanteerd door niet representative testdata, volgens het principe garbage in, garbage out. 
	
	\section{Mean-Squared Error}
	De MSE is een algemene maat voor hoe accuraat het model in het voorspellen van de klas tot waar een bepaalde test
	waarde tot behoort. De MSE kan op veel verschilende manieren worden berekent de meest courante is.
	\begin{equation}
		\(MSE(\hat{\theta}) = Var\hat{\theta} + (bias \hat{\theta})^2\)
	\end{equation}
	Waarbij \(\hat{\theta}\) de verwachte waarde volgens de schatter is. Een lager MSE is steedts beter. als er geen bias is, is de variantie gelijk aan de MSE.  
	Hier wordt dus bijna altijd naar gerefereerd om te bepalen welk model beter is.
	
	
	
\end{document}

%bronnen:
% ISL

