\documentclass[TeamE-eindrapport]{subfiles}

\begin{document}
	
	\chapter{Supervised Learning}
	
	\label{tekst:supervisedlearning}
	
	\section{Supervised vs unsupervised}
	
	Machine learning kan worden opgesplitst in twee grote categorieën; \textit{supervised learning} enerzijds en \textit{unsupervised learning} anderzijds. De naam van beide vormen geeft eigenlijk al weg wat deze vormen precies inhouden.
	
	Bij \textit{unsupervised learning} hebben we voor elk datapunt in onze dataset zowel features als outputs voor handen. Er is sprake van zogenaamde 'input-outputparen'. We kunnen ons model trainen op het vinden van een verband tussen de \(x\)-waarden en de \(y\)-waarden in onze trainingsdataset. Het uiteindelijke doel is dan dat het model voor een nieuwe dataset, op basis van de features in die dataset, een output zal voorspellen.
	
	Bij \textit{unsupervised learning} zal de data enkel uit inputs of \(x\)-waarden bestaan en hebben we dus geen afhankelijke variabelen. Een \textit{unsupervised learning}-model zal zelf patronen of structuren trachten te vinden in de data en vervolgens de data zelf onderverdelen in klassen. De details achter \emph{unsupervised learning} liggen buiten de scope van dit rapport en zullen we dus achterwege laten.
	
	\section{Vormen van supervised learning}
	
	\textit{Supervised learning} kan nog verder gesplitst worden in twee categorieën; de twee voornaamste vormen van \textit{supervised learning} zijn regressie en classificatie. 
	
	Bij regressie zijn de outputs altijd getallen. Deze numerieke outputs kunnen eender welke waarde aannemen. We kunnen hierbij dus spreken van \textit{kwantitatieve data}. Het getrainde model zal trachten een continu numeriek resultaat te voorspellen voor de output van een nieuw datapunt - waarop het model dus niet getraind is.
	
	Dit staat in tegenstelling met classificatie, waar de outputs enkel een telbaar aantal waarden kunnen aannemen. We zullen in hoofdstuk \ref{tekst:classificatie} classificatie in meer detail toelichten.
	
\end{document}