\documentclass[TeamE-eindrapport]{subfiles}

% Marie haar deel

\begin{document}
		
	Variantie in het kader van machine learning is de maat voor de gevoeligheid van een model. Het geeft inzicht over de flexibiliteit van      het model, met name hoe goed het model zich kan aanpassen aan verschillende datasets. De variantie analyseert het verschil tussen de        door het model voorspelde waarde en de werkelijke waarde van een variabele. Voor het berekenen van de variantie wordt het verschil          tussen de werkelijke waarde en de voorspelde waarde genomen en dit vervolgens gekwadrateerd. Neem een variabele \(x\), dan is \(f(x)\)      de waarde horende bij de variabele \(x\) en \(f(\hat{x})\) de voorspelde waarde voor \(x\). Dan kan de variantie als volgt worden           gedefinieerd:
	\[Var (x)= (E(f(x)- f(\hat{x}))^2\]
	Om een goed model te verkrijgen is het wenselijk om de variantie zo laag mogelijk te nemen. Een lage variantie betekent dat het model       weinig afhankelijk is van veranderingen in de trainingsset en minder gevoelig is voor uitzonderlijke waarden binnen de set.  Hierdoor       zullen er accuratere voorspellingen gedaan worden bij andere datasets.
	Bij een hoge variantie past het model zich nauwkeurig aan de trainingsset aan, maar zal het model de eventuele uitschieters ook als         belangrijk beschouwen. Wanneer de trainingsset verandert, zal het model dus ook duidelijke aanpassingen vertonen.  Bij het gebruik van      een andere dataset zal er dan  een duidelijker verschil zijn tussen de werkelijke en de voorspelde waarden. Een toename van de              variantie zal leiden tot een verminderde nauwkeurigheid van het model.
\end{document}